\documentclass{assignment-373}
\usepackage{mathabx}
\usepackage{tikz}
\usepackage{color}


\anum{2}
\course{CSC373}
\duedate{Nov 14, 2022}
\filename{ps2.pdf, ps2.tex, likeliest\_evolution.py}

\begin{document}

\think

\textbf{Please see the course information sheet for the late submission
  policy.}

\begin{enumerate}
\item \textbf{[15 points]}

  You are planning a new office for the tech giant
  \textsc{SpaceNook}. You know that one important aspect to keeping
  your employees happy and functioning well is quick access to free
  snacks.  You will determine amongst all the seating locations for
  employees in the new office, what is the longest time it would take
  for an employee to reach their nearest snack station.

%   You are studying the layout of the city of \textsc{Mediviala}.
%   The city is
% filled with several historical buildings. Being medivial, the dominant
% material used for constructing these buildings is wood. This makes
% protecting these buildings from fire a very important aspect of city
% planning. We will determine that among all the historical buildings in
% the city, what was the longest time it would have taken in order to
% get help in case of a fire in one of these buildings.

  You have access to a possible map for the new office for
  \textsc{SpaceNook}.  All the employee locations in the new office
  have been numbered from $[n] = \{1,\ldots, n\}.$
  %
  
  You are given a list $P$ of pathways in the office (including
  hallways, stairs, elevators), with $|P| = m$. Each such pathway
  connects two end points that are among the office locations numbered
  $1, \ldots, n,$ and is specified as a pair of such locations.
%
  For each pathway, you are also provided the time in minutes required
  to travel along the pathway in either direction.

% $R$ connecting various locations in the city, along with the time
% required to travel along them. Concretely, each road connects two end
% points that among the city locations numbered $1, \ldots, n.$ For each
% road, you are also given the time in minutes taken to travel along
% that road. Note that a road can be traveled in either direction.

% with $n$ locations
% $[n] := \{1,\ldots,n\}$, where any two locations $i$ and $j$ are
% potentially connected by a bidirecitonal road of length $\ell(i,j)$,
% where $i, j \in [n]$. Let $m$ denote the number of the connections
% among the locations in the city.
  Finally, you are given two lists. The first list $E \subseteq [n]$
  contains the locations of all the employees in the office.
%
  The second list $S \subseteq [n]$ represents the list of snack
  stations in the office.
% are also
% given two sets with $H,F \subset [n]$ and $H \cap F = \emptyset$,
% where $H$ and $F$ represent the historical buildings and the fire
% stations in the city, respectively.

  Your goal is to determine, the shortest distance for each employee
  location, to its \emph{closest} snack station.
%
  Follow the steps below to design an algorithm for this problem that
  has a worst-case time complexity of $O(m \log n).$
%
\begin{enumerate}
\item \textbf{(1 point)} Describe how will you model the problem, and
  the data-structure(s) that you will use to store the input
  information.
\item \textbf{(2 points)} Briefly describe in up to 3 lines how you
  can solve the problem in $O\left(|E|m \log n\right)$ worst-case
  time, where $|E|$ denotes the number of employee locations in the
  office.
\item \textbf{(2 points)} Briefly describe in up to 3 lines how you
  can solve the problem in $O\left(|S|m \log n\right)$ worst-case
  time, where $|S|$ denotes the number of snack stations in the
  office.
\item \label{item:algo} \textbf{(5 points)} Propose an algorithm that
  solves the problem in $O(m \log n)$ worst-case time, and justify its
  correctness.
\item \textbf{(3 points)} Write the pseudo-code for the final
  algorithm described in part~\ref{item:algo}
  % \item \textbf{(5 points)} Give a correctness proof for your algorithm.
\item \textbf{(2 points)} Describe a short modification to the above
  algorithm so that it also finds the closest snack station for each
  employee location in addition to the shortest distance in
  $O(m \log n)$ worst-case time.
\end{enumerate}


\item \textbf{[15 points]}

You work for a national biotechnology lab, and are studying the
evolution of the SARS-CoV-2 virus (also referred to as Covid-19).
You are trying to determine the most likely way the current variant
evolved from the original variant.

You have a database of genomic sequences of lots of variants of the
virus. You have access to a software that predicts the probability of
a variant \textsf{Var1} evolving into another variant
\textsf{Var2}. However, this software only works accurately for a
small set of the pairs of variants that have similar genomes. These
probabilities are given to you as tuples in a list,
e.g. \textsf{[(Var1, Var2, 0.95), (Var2, Var1, 0.95), (Var1, Var3,
  0.90), (Var2, Var4, 0.85). (Var3, Var4, 0.90)]}. The tuple
\textsf{(Var1, Var2, 0.95)} indicates that there is a 0.95 probability
of \textsf{Var1} evolving into \textsf{Var2}.

For the purpose of this problem, we will assume that all the
probabilities given in the above list are accurate, and that if a pair
is not in the list, that evolution cannot happen directly (but can
happen via intermediate variants). In the above example, \textsf{Var1}
cannot directly evolve into \textsf{Var4}, but can indirectly evolve
into \textsf{Var4} (via either \textsf{Var2} or \textsf{Var3}).

Your goal is to design and implement an algorithm to find a sequence
of possible evolutions starting from a given starting variant,
e.g. \textsf{Var1}, to the final variant, e.g. \textsf{Var4}, that
maximizes the probability of evolution, together with the probability
of this sequence of evolutions.

Assume that different jumps are independent, and hence the outcome of
one evolution does not affect the probabilities of other evolutions.

In the above example, the sequence of evolutions \textsc{Var1} $\to$
\textsc{Var3} $\to$ \textsc{Var4} has a probability of 0.90*0.90 =
0.81. This is larger than the probability of 0.95*0.85 = 0.8075 for
the sequence of evolutions \textsc{Var1} $\to$ \textsc{Var2} $\to$
\textsc{Var4}.

\textbf{(12 points)} Implement the function
\texttt{likeliest\_evolution} which has the following input
parameters:
\begin{itemize}
\item \texttt{num\_variants}: an integer $n \ge 0$ specifying the
  number of variants being considered. (For simplicity, you can assume
  that the variants are numbered $1,\ldots,n$ rather than using strings)
\item \texttt{poss\_evolutions}: a list of triples, where each triple
  $(i,j,p)$ represents a possible evolution from variant $i$ to
  variant $j,$ along with the probability $p$ of transition.
\item \texttt{s}: an integer $s$ representing the initial variant
\item \texttt{t}: an integer $t$ representing the final variant
\end{itemize}





Your algorithm should output:
\begin{itemize}
\item a list of integers representing the sequence of variant
  evolution that maximizes the probability of evolution of the final
  variant, starting from the initial variant.
\item a float representing the total probability of evolution along
  the above sequence.
\end{itemize}



        \textbf{Please include comments in your code to explain what
          your algorithm is doing.}


\textbf{(3 points)} Clearly and concisely explain your approach for
the above algorithm.

For the above example, a call to \texttt{likeliest\_evolution} looks
as follows:
\begin{verbatim}
likeliest_evolution(
 4,
 [[1,2,0.95], [2,1,0.95], [1,3,0.90], [2,4,0.85], [3,4,0.90]],
 1, 4)
\end{verbatim}
The output should the sequence of evolution \texttt{[1,3,4]} and the
probability of this sequence as \texttt{0.81}.


Requirements:
\begin{itemize}
\item Your code must be written in Python 3, and the filename must be \texttt{likeliest\_evolution.py}.
\item We will grade only the \texttt{likeliest\_evolution} function;
  please do not change its signature in the starter code. You may
  include as many helper functions as you wish
\item You are not allowed to use the built-in Python dictionary or set
  or any other built-in data-structure (other than a list or array).
\item To get full points, your algorithm must have an asymptotic
  worst-case time complexity of $O(|T| \log n),$ where $T$ is the size
  of the list \texttt{poss\_evolutions}.
\item For each test-case that your code is tested on, your code must
  run within 10x the time taken by our solution. Otherwise, your code
  will be considered to have timed out.
\end{itemize}



{\color{red}



  }
\item \textbf{[15 points]}

You are working for an electric self-driving taxi startup in Toronto.
 % their team responsible for managing driving routes for your taxis.
%
In order to determine the minimum battery capacity your taxis need to
have, you want to compute the maximum time taken by a taxi to travel
between any source location and any destination location in the city.

We're going to build the first basic model to estimate this time.
%
For simplicity, we will assume that the city can be described as an
$n \times n$ grid, aligned with the cardinal directions (North, South,
East, West), where $(0,0)$ is the South-West corner of the city.
%
You are given an $n \times n$ array $M$ such that $M[i][j]$
corresponds to the $(i,j)$ square in the grid for $0 \le i, j < n.$
%
Every entry $M[i][j]$ is either 1 or 0, where a 1 indicates that the
$(i,j)$ square in the city is a 'road', and a taxi can be present in
this square at any time. For simplicity, we will assume that on a
square representing a road, the car can potentially be facing any of
the four directions (North, South, East, West).
%
$M[i][j]$ being 0 indicates an obstacle at the square $(i, j)$, and a
taxi can never be present in such a square.

For this basic model, we will assume that the taxi can execute two
kinds of instructions,
\begin{enumerate}
\item Move straight ahead by $k$ squares (for some $k$).

  The time taken to move straight ahead by $k$ squares is $(a*k + b)$
  minutes where $a, b$ are fixed positive constants that are provided
  as part of the input.
% 
  This models the fact that starting and stopping the car takes some
  extra time (specified by $b$), in addition to the time taken to move
  per square (specified by $a$).
  %
  A taxi can execute such an instruction only if the $k$ locations
  straight ahead are roads and not obstacles.

\item Turn right or left within the same square by 90 degrees.

  The time taken to take a single 90 degree turn is $c$ minutes (a
  non-negative constant specified as part of the input). Note that the car
  can take a U-turn by taking 2 consecutive turns.
\end{enumerate}
%
%

Your goal is to determine a starting square along with a starting
orientation (N, S, E, W), and a finishing square along with a
finishing orientation, such that the shortest time to go from the
starting configuration (location and facing direction) to the
finishing configuration is the largest possible over all possible
valid choices of starting and finishing configurations.
%
Note that the taxi can never be on a square representing an obstacle,
and can never leave the grid.

For example, consider the $3 \times 3$ grid, where black indicates an
obstacle, and a white square indicates a road. 
\begin{center}
\includegraphics[scale=0.5]{city-layout.png}
\end{center}
For $a=b=c=1,$ the maximum possible time taken by the taxi would be 11
minutes. One possible starting and ending configuration that requires
11 minutes is to go from the starting square $(1,2),$ where at the
beginning the taxi is facing North, to go to the final location $(1,0)$
with the taxi facing North at the final location. For this, the
fastest sequence of steps for the taxi would be as follows:
\begin{enumerate}
\item Turn right (takes 1 minute, the taxi is now at $(1,2)$ and faces
  East)
\item Go ahead 1 steps (takes 2 minutes, the taxi is now at $(2,2)$
  and faces East)
\item Turn right (takes 1 minute, the taxi is now at $(2,2)$ and faces
  South)
\item Go ahead 2 steps (takes 3 minutes, the taxi is now at $(2,0)$
  and faces South)
\item Turn right (takes 1 minute, the taxi is now at $(2,0)$ and faces
  West)
\item Go ahead 1 steps (takes 2 minutes, the taxi is now at $(1,0)$
  and faces West)
\item Turn right (takes 1 minute, the taxi is now at $(1,0)$ and faces
  North)
\end{enumerate}


% For example, given a directed graph $G=(V,E)$, with
% \begin{align*}
%   & V=\{1,2,3,4\},~~E=\{(1 \to 2), (1 \to 3), (1 \to 4), (2 \to 4), (3
%   \to 4)\}, \\
%   & p((1 \to 2)) = 0.2, p((1 \to 3)) = 0.5, p((1 \to 4)) = 0.1, p((2 \to
%   4)) = 0.5, p((3 \to 4)) = 0.3,\\
%   & s=1, t=4,
% \end{align*} the $(s,t)$ path that maximizes the
% probability of a successful traversal is $(1 \to 3 \to 4)$ achieving a
% probability
% of $0.5 \times 0.3 = 0.15$.

% For example, given a directed graph $G=(V,E)$, with  \[ V=\{1,2,3,4\},~~E=\{(1,2,0.2), (1,3, 0.5), (1,4,0.1), (2,4,0.5), (3,4,0.3)\},~~s=1, t=4, \]
% the $(s,t)$ path that maximizes the probability of a successful traversal is $(1,3,4)$ achieving a weight of $0.5 \times 0.3 = 0.15$. 


Design an algorithm which takes the following inputs as parameters:
\begin{itemize}
\item \texttt{grid\_size}: an integer $n \ge 0$ specifying the size
  of the grid.
\item \texttt{a}: a number $a \ge 0$
\item \texttt{b}: a number $b \ge 0$
\item \texttt{c}: a number $c \ge 0$
\item \texttt{M}: A 2-dimensional $n \times n$ array $M$ specifying
  the layout of the city.
\end{itemize}

Your algorithm should return the maximum possible time taken by the
taxi to go from a starting configuration to an ending configuration
using the fastest possible sequence of steps, where the maximum is
taken over all possible feasible configurations.

We will model this as a graph problem.  If necessary, you are allowed
to invoke algorithms covered in class without describing how they
work, but you must clearly describe the inputs provided to the
algorithms. e.g. If invoking a graph algorithm, clearly describe the
graph: what are the vertices, edges (undirected/directed), edge
weights (if any) for the graph? Also describe any other inputs to the
algorithm, e.g. starting vertex for Dijkstra's algorithm.

You are also free to design your own algorithm from scratch.

\begin{enumerate}
\item (7 points) Design an $O(n^6)$ time algorithm for this
  problem. Explain clearly why your graph models the problem
  correctly, and why your algorithm returns the correct answer.
  
\item (8 points) Consider the special case of $b=0.$ (This means that
  we ignore the time taken by the taxi to start/stop, and hence ``Move
  straight by k steps'' is the same as ``Move straight by 1 step''
  executed $k$ times in sequence).

  Design an $O(n^4\log n)$ time algorithm for the problem in this
  special case. Explain clearly why your graph models the problem
  correctly, and why your algorithm returns the correct answer.
\end{enumerate}

\end{enumerate}

\end{document}

%%% Local Variables:
%%% mode: latex
%%% TeX-master: t
%%% End:
