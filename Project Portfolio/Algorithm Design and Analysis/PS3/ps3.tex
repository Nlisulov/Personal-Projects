\documentclass{assignment-373}
\usepackage{mathabx}
\usepackage{tikz}
\usepackage{color}


\anum{3}
\course{CSC373}
\duedate{Dec 9, 2022}
\filename{ps3.pdf, ps3.tex}

\begin{document}

\think

\textbf{Please see the course information sheet for the late submission
  policy.}

\begin{enumerate}
\item \textbf{[15 points]}
  In this problem, we will show that the maximum-flow problem is
  equivalent to the max-single-edge-flow problem.

  Recall that in a max-flow instance, we require that for every
  vertex, the in-flow is the same as out-flow, except the source ($s$)
  and target ($t$) vertices.

  In the \textbf{max-single-edge-flow} problem, we are given a
  directed graph $G(V, E)$ with capacities $c_e > 0$ on edges
  $e \in E,$ and a special (directed) edge $e^{\star} \in E.$ Our goal
  is to find a flow $f_e$ for $e \in E,$ that satisfies the following
  constraints:
  \begin{itemize}
  \item $f$ satisfies the direction and capacity constraints for all
    edges, i.e., $\forall e \in E, 0 \le f_e \le c_e,$ 
  \item $f$ satisfies in-flow equals out-flow for \textit{all}
    vertices, i.e.,
    \[ \forall v \in V, \quad \sum_{u: (u \to v) \in E} f_{(u \to v)}
      = \sum_{u: (v \to u) \in E} f_{(v \to u)},
    \]
  \end{itemize}
  and subject to the above constraints, our goal is to maximize the
  flow on the special edge $e^{\star}.$

  \begin{enumerate}
  \item (6 points) Give a reduction from the {max-single-edge-flow}
    problem to the max-flow problem. That is, given an instance of
    max-single-edge-flow, construct a max-flow instance such that
    solving the max-flow instance allows you to solve the
    max-single-edge-flow instance.  Justify both sides of the
    reduction as done in class.

    Your algorithm for constructing the max-flow instance must run in
    $O(|V|+|E|)$ time, where $G(V, E)$ is the graph for the
    max-single-edge-flow instance.
    
  \item (9 points) Give a reduction from the max-flow problem to the
    {max-single-edge-flow} problem. That is, given an instance of
    max-flow, construct an instance of the max-single-edge-flow
    problem such that solving the {max-single-edge-flow} instance
    allows you to solve the max-flow instance. Justify both sides of
    the reduction.

    Your algorithm for constructing the max-single-edge-flow instance
    must run in $O(|V|+|E|)$ time, where $G(V, E)$ is the graph for
    the max-flow instance.
  \end{enumerate}
  
\item \textbf{[15 points]}
  The city of Toronto wants to monitor all traffic that enters Toronto
  from Mississauga, and leaves Toronto towards Scarborough. It is
  planning to do so via cameras mounted at select intersections
  throughout the city.

  We are going to model this as an instance of the \textbf{Undirected
    Bottleneck Problem}. The road network of Toronto is given as an
  undirected graph $G(V, E)$, with vertices $V$ representing
  intersections, and edges $E$ representing roads connecting these
  intersections. We assume for simplicity that a car can only travel
  along a sequence of roads, and every road can be traversed in either
  direction, i.e., the edges are undirected. Say $|V|=n,$ and $|E|=m.$

  We are given 2 subets of these intersections.  $M \subseteq V$ is
  the set of intersections where traffic from Mississauga can enter
  Toronto. Similarly, $S \subseteq V$ is the set of intersections
  where traffic can leave Toronto for Scarborough.

  Given $G(V, E)$ and $M, S,$ the goal of the \textbf{Undirected
    Bottleneck Problem} is to determine a \textit{bottleneck} set
  $C \subseteq V$ of intersections with the smallest possible
  cardinality where cameras can be mounted (one per such
  intersection), such that every car starting from Mississauga, i.e.,
  from some intersection in $M,$ traveling along some sequence of
  roads, and leaving Toronto at some intersection in $S,$ must go
  through one of the intersections in $C.$

  \begin{enumerate}
  \item (10 points) Give a reduction from this problem to
    maximum-flow, i.e. construct a max-flow instance such that running
    a max-flow algorithm (such as the Ford-Fulkerson algorithm),
    allows us to find the minimum number of checkpoints needed. Your
    construction must run in $O(m+n)$ time.

    Fully justify your reduction. Remember that a reduction consists
    of 2 parts. The first part involves showing that if every
    bottleneck in our \textbf{Undirected Bottleneck Problem} has large
    cardinality, then the resulting max-flow instance has a flow with
    large flow-value.

    For the second part, show that any flow in the max-flow instance
    with large flow value implies that every bottleneck set in the
    original Undirected Bottleneck Problem instance has a large
    cardinality.
    
  \item (5 points) Say we use the Ford-Fulkerson algorithm to solve
    the max-flow instance constructed in part a. Using the max-flow
    returned by the Ford-Fulkerson algorithm, and an additional
    $O(m+n)$ time at most, find a set of checkpoints $C \subseteq V$
    that is a solution to the \textbf{Undirected Bottleneck Problem}
    with minimum cardinality.
  \end{enumerate}
  
\item \textbf{[15 points]}
%
  You are organizing a hackathon at the Department of Mathematical \&
  Computational Sciences (MCS).  Students have organized themselves
  into teams of sizes between 4 and 6.  Note that each student can
  participate in multiple teams. Assume that there are a total of $n$
  students, and $t$ teams.

  You wish to identify a set $C$ of student representatives. We want
  the set of representatives $C$ to contain at least one student from
  each of the $t$ teams across the department. Note that one student
  representative could represent multiple teams. Our goal is to find a
  set of representatives $C$ with the smallest possible size.
  
  \begin{enumerate}
  \item (3 points) Design an approximation algorithm for this problem
    that runs in time polynomial in $n, t$ and achieves an
    approximation ratio of $O(\log n)$ for the problem of finding the
    smallest set of representatives. Justify the correctness of your
    algorithm, its running time and approximation ratio guarantees.
  \item (12 points) Design another approximation algorithm for this
    problem that runs in time polynomial in $n, t$ and achieves an
    \textbf{approximation ratio of 6} for the problem of finding the
    smallest set of representatives. Justify the correctness of your
    algorithm, its running time and approximation ratio
    guarantees. (Hint: consider an algorithm like GreedyVertexCover).
  \end{enumerate}

\end{enumerate}

\end{document}

%%% Local Variables:
%%% mode: latex
%%% TeX-master: t
%%% End:
